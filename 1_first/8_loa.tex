
%
%\addcontentsline{toc}{chapter}{\, \, \, LIST OF ABBREVIATIONS}{\pageref{LOA}}
%
%\label{LOA}
%
%%% The following should contain the list of all abbreviations to be used in the Thesis write-up. Enter the
%%% the abbreviations and their full form as shown below for example.
%
%\nomenclature{LED}{Light Emitting Diode}
%\nomenclature{MOS}{Metal Oxide Semiconductor}
%\nomenclature{PC}{Personal Computer}
%
%\printnomenclature [2.5 cm]
%to use the glossaries package, the symbols and acronyms have to be defined first
%SYMBOLS: Symbols can be defined using the following code
\newcommand{\sym}[4]{%This command can be used to define a symbol as explain below
	\newglossaryentry{#1}{%
		name={#2},
		description={#3},
		symbol={#2},
		sort={#4}
	}%
}%
%The format for defining a symbol is as follows
%\sym{label}{symbol}{description}{alphabetical sorting name}
%label: This will be used later on in the main body to use the symbol
%symbol: The actual symbol to be used
%description: The description of the symbol
%alphabetical sorting name: This name will be used to sort the symbol in alphabetical order if desired
%example symbol definition
\sym{mysymbol}{\ensuremath{\Gamma}}{This is a symbol}{gamma}
\sym{yoursymbol}{\ensuremath{\Theta}}{This is a symbol}{tetha}

%An acronym or abbreviation can be defined as follows
%\newacronym{label}{short}{long}
%label: this will be used to call the acronym in the main body
%short: the acronym
%long: the full form of the acronym
%example acronym
\newacronym{phd}{Ph.D.}{Doctor of Philosophy}
\newacronym{uav}{UAV}{Unmanned Aerial Vehicle}
\newacronym{vrp}{VRP}{Vehicle Routing Problem}

\makeglossaries